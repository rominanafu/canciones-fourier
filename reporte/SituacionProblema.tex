\documentclass[12pt, letterpaper]{article}
\usepackage{graphicx}
\usepackage{hyperref}
\usepackage{amssymb}
\usepackage{amsmath}
\usepackage{float}
\usepackage{mathtools}
\usepackage{enumitem}
\usepackage[margin=1in]{geometry}
\usepackage[figurename=Figura]{caption}

\title{%
  Situación Problema: Análisis de Audio usando Fourier \\
  \large F1009: Análisis de métodos matemáticos para la física}

\begin{document}

\maketitle

\begin{tabular}{ccc}
Juan Pablo Guerrero Escudero & Romina Nájera Fuentes & Juan Braulio Olivares Rodríguez
\end{tabular}

\section{Introducción}

\section{Teoría}

\subsection{Conceptos físicos}

\subsection{Análisis de las canciones}

Para la clasificación de las canciones entre los géneros de música instrumental o reggaetón,
realizaremos un análisis espectral, el cual busca descomponer una serie de tiempo en las
ondas senoidales que la conforman \cite{Montenegro-2009}. Este análisis permitirá obtener las
diferentes frecuencias que conforman al pedazo de canción a analizar, y poder sacar conclusiones
sobre el género de la canción. \medskip

\noindent Para ello, utilizaremos la transformada de Fourier, utilizada
comúnmente en el campo científico, como en la acústica y el procesamiento de señales.
Esta herramienta transforma el dominio de una señal, pasando del tiempo a la frecuencia,
sin alterar su contenido \cite{Bernal-1999}.
Al perderse la noción del tiempo, analizaremos los rangos de frecuencias en los
que se encuentran magnitudes más grandes, para así identificar si la canción presentada
es instrumental o reggaetón.

\subsubsection{Transformada de Fourier}

Por definición, la transformada de Fourier de una función $f(x)$ es dada por la ecuación \ref{eq:fourier}
\begin{align}
	\phi_f(\alpha) &= \int_{-\infty}^{\infty} e^{i\alpha x} f(x) dx
	\label{eq:fourier}
\end{align}

\noindent Esta transformada fue desarrollada por Jean-Baptiste Joseph Fourier en el siglo XIX,
quien inició proponiendo que cualquier función arbitraria de una variable
podía ser expresada como una combinación lineal de funciones de senos y cosenos,
que son las series de Fourier \cite{OGorman-2023}. \medskip

\noindent A través de esas series, logró sintetizar la transformada, la cual
es distinguida por:
\begin{itemize}
  \item Determinar qué frecuencias están presentes en una señal.
  \item Transformar una señal del dominio temporal al dominio de frecuencia y viceversa.
\end{itemize}

\noindent Así como las series de Fourier descomponen una función en senos y cosenos, la
transformada descompone una señal en sus frecuencias, y aquellas que tengan una mayor
amplitud, se verán representadas como picos más altos, así como se puede observar en
la figura \ref{fig:fourier}.

\begin{figure}[H]
  \centering
  \includegraphics[width=0.9\textwidth]{FourierSeries_Freq.png}
  \caption{Representación visual de la transformada de Fourier \cite{OGorman-2023}.}
  \label{fig:fourier}
\end{figure}

\subsubsection{Espectrogramas}



\subsubsection{Rangos de frecuencias}



Conceptos de física relevantes:
\begin{enumerate}
    \item Ondas de sonido
    \item Frecuencias de audio/sonido
    \item Sonidos armónicos
    \item Beats
\end{enumerate}

\noindent Análisis matemático + fundamentos:
\begin{enumerate}
    \item Análisis espectral de canciones
    \item Transformada de Fourier
    \item Identificación de reggaeton/instrumental
\end{enumerate}

\section{Resultados}

\section{Conclusiones}



\begin{thebibliography}{9}
  \bibitem{university-physics}
  H. D. Young and Roger A. Freedman, \emph{University Physics with Modern Physics}, 
Addison-Wesley, San Francisco, 2012. %%Para citar \cite{university-physics}
  \bibitem{Montenegro-2009}
  A. Montenegro, \textit{CORE}, 2009, \url{https://core.ac.uk/download/pdf/6448967.pdf}
  \bibitem{Bernal-1999}
  J. Bernal, P. Gómez y J. Bobadilla, \textit{Estudios de fonética experimental}, 1999, \textbf{10}, 75-105.
  \bibitem{OGorman-2023}
  L. O'Gorman, \textit{DIBS Methods Meetings}, 2023, \url{https://dibsmethodsmeetings.github.io/fourier-transforms/}
  
  \end{thebibliography}
\end{document}
