\documentclass[12pt, letterpaper]{article}
\usepackage{graphicx}
\usepackage{hyperref}
\usepackage{amssymb}
\usepackage{amsmath}
\usepackage{float}
\usepackage{mathtools}
\usepackage{enumitem}
\usepackage[margin=1in]{geometry}
\usepackage[figurename=Figura]{caption}

\title{%
  Situación Problema: Análisis de Audio usando Fourier \\
  \large F1009: Análisis de métodos matemáticos para la física}

\begin{document}

\maketitle

\begin{tabular}{ccc}
Juan Pablo Guerrero Escudero & Romina Nájera Fuentes & Juan Braulio Olivares Rodríguez
\end{tabular}

\section*{Introducción}

\section*{Teoría}

Conceptos de física relevantes:
\begin{enumerate}
    \item Ondas de sonido
    \item Frecuencias de audio/sonido
    \item Sonidos armónicos
    \item Beats
\end{enumerate}

\noindent Análisis matemático + fundamentos:
\begin{enumerate}
    \item Análisis espectral de canciones
    \item Transformada de Fourier
    \item Identificación de reggaeton/instrumental
\end{enumerate}



\section*{Resultados}

\section*{Conclusiones}

\section*{Referencias}

\end{document}
